\documentclass[12pt,twoside]{article}
%\usepackage[table]{xcolor} % important to avoid options clash.
%\input{02465shared_preamble}
%\usepackage{cleveref}
\usepackage{url}
\usepackage{graphics}
\usepackage{multicol}
\usepackage{rotate}
\usepackage{rotating}
\usepackage{booktabs}
\usepackage{hyperref}
\usepackage{pifont}
\usepackage{latexsym}
\usepackage[english]{babel}
\usepackage{epstopdf}
\usepackage{etoolbox}
\usepackage{amsmath}
\usepackage{amssymb}
\usepackage{multirow,epstopdf}
\usepackage{fancyhdr}
\usepackage{booktabs}
\usepackage{xcolor}
\newcommand\redt[1]{ {\textcolor[rgb]{0.60, 0.00, 0.00}{\textbf{ #1} } } }


\newcommand{\m}[1]{\boldsymbol{ #1}}
\newcommand{\yoursolution}{ \redt{(your solution here) } } 



\title{ Report 1 hand-in }
\date{ \today }
\author{Alice (\texttt{s000001})\and  Bob (\texttt{s000002})\and Clara (\texttt{s000003}) } 

\begin{document}
\maketitle

\begin{table}[ht!]
\caption{Attribution table. Feel free to add/remove rows and columns}
\begin{tabular}{llll}
\toprule
                                                        & Alice   & Bob    & Clara   \\
\midrule
 1: A basic blaster-business                            & 0-100\%  & 0-100\% & 0-100\%  \\
 2: Warmup                                              & 0-100\%  & 0-100\% & 0-100\%  \\
 3: Manually computing $J_{N-1}$                        & 0-100\%  & 0-100\% & 0-100\%  \\
 4: Compute optimal policy and value function           & 0-100\%  & 0-100\% & 0-100\%  \\
 5: Kiosk2                                              & 0-100\%  & 0-100\% & 0-100\%  \\
 6: Explaining the policy                               & 0-100\%  & 0-100\% & 0-100\%  \\
 7: Policy explanation continued                        & 0-100\%  & 0-100\% & 0-100\%  \\
 8: Go east                                             & 0-100\%  & 0-100\% & 0-100\%  \\
 9: Describe the go-east problem                        & 0-100\%  & 0-100\% & 0-100\%  \\
 10: Predict consequence of actions                     & 0-100\%  & 0-100\% & 0-100\%  \\
 11: Possible future states                             & 0-100\%  & 0-100\% & 0-100\%  \\
 12: Shortest path                                      & 0-100\%  & 0-100\% & 0-100\%  \\
 13: Predict consequence of actions with one ghost      & 0-100\%  & 0-100\% & 0-100\%  \\
 14: Possible future states with one ghost              & 0-100\%  & 0-100\% & 0-100\%  \\
 15: Optimal one-ghost planning                         & 0-100\%  & 0-100\% & 0-100\%  \\
 16: Predict consequence of actions with several ghosts & 0-100\%  & 0-100\% & 0-100\%  \\
 17: Future states                                      & 0-100\%  & 0-100\% & 0-100\%  \\
 18: Optimal planning                                   & 0-100\%  & 0-100\% & 0-100\%  \\
\bottomrule
\end{tabular}
\end{table}

%\paragraph{Statement about collaboration:}
%Please edit this section to reflect how you have used external resources. The following statement will in most cases suffice: 
%\emph{The code in the irls/project1 directory is entirely}

%\paragraph{Main report:}
Headings have been inserted in the document for readability. You only have to edit the part which says \yoursolution. 

\section{The kiosk (\texttt{kiosk.py})}
\subsubsection*{{\color{red}Problem 1:  A basic blaster-business}}

\yoursolution 	
\redt{To get you started: \begin{align}
	N & = 14 \\
	\mbox{for $k=0,\dots,N$: }\quad	\mathcal{S}_k & = \dots \\
	\mbox{for $k=0,\dots,N-1$: }\quad \mathcal{A}_k(x_k) & = \dots \\
	 & \vdots 
\end{align} }

\subsubsection*{{\color{red}Problem 3:  Manually computing $J_{N-1}$}}
		
	\yoursolution 	
	$$
	J_{N-1}(20)  = ...
	$$
	
\subsubsection*{{\color{red}Problem 6:  Explaining the policy}}

					The first policy... this can be explained by noting ... \yoursolution 

\subsubsection*{{\color{red}Problem 7:  Policy explanation continued}}
	
	$$\mu_{N-1}(0) = ...$$
\yoursolution 		

\section{Avoid the droid (\texttt{pacman.py)}} 
\subsubsection*{{\color{red}Problem 9:  Describe the go-east problem}}
	
		The environment is an example of a .... \\		
		The controller is an example of a ...
		\yoursolution 	
	
\end{document}